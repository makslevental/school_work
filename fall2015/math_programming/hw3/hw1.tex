%% LyX 2.1.3 created this file.  For more info, see http://www.lyx.org/.
%% Do not edit unless you really know what you are doing.
\documentclass[oneside]{amsart}
\usepackage[latin9]{inputenc}
\setlength{\parskip}{\medskipamount}
\setlength{\parindent}{0pt}
\usepackage{mathtools}
\usepackage{amsthm}
\usepackage{amstext}
\usepackage{amssymb}
\usepackage{cancel}

\makeatletter
%%%%%%%%%%%%%%%%%%%%%%%%%%%%%% Textclass specific LaTeX commands.
\numberwithin{equation}{section}
\numberwithin{figure}{section}
  \theoremstyle{plain}
  \newtheorem*{lem*}{\protect\lemmaname}
  \theoremstyle{plain}
  \newtheorem*{cor*}{\protect\corollaryname}

%%%%%%%%%%%%%%%%%%%%%%%%%%%%%% User specified LaTeX commands.

%
\usepackage{amsfonts}
\usepackage{xfrac}
%\usepackage{mathabx}
\usepackage{nopageno}%%%  The following few lines affect the margin sizes. 
\usepackage{bm}
\addtolength{\topmargin}{-.5in}
\setlength{\textwidth}{6in}       
\setlength{\oddsidemargin}{.25in}              
\setlength{\evensidemargin}{.25in}         
  
\setlength{\textheight}{9in}
\renewcommand{\baselinestretch}{1}
\reversemarginpar   
%
%

\makeatother

  \providecommand{\corollaryname}{Corollary}
  \providecommand{\lemmaname}{Lemma}

\begin{document}

\title{ESI 6420 Homework 3 Solutions}


\author{Maksim Levental}


\date{\today}

\maketitle
Time spent: 15 hours

Collaborators: Chris Gianelli
\begin{enumerate}
\item [1.1] The set is not convex: take as a counterexample the $T=\left\{ \left(0,1\right)\right\} $
and the set $S=\left\{ \left(0,0\right),\left(0,2\right)\right\} $.
Then the set of points closer to $S$ is $\left(-\infty,.\frac{1}{2}\right)\cup\left(\frac{3}{2},\infty\right)$,
which is clearly not convex.
\item [1.2] The set is convex. Let 
\[
S=\left\{ x\in\mathbb{R}^{n}\big|x+S_{2}\subset S_{1}\right\} 
\]
and $x_{1},x_{2}\in S$. Fix an arbitrary $s_{2}\in S_{2}$. Since
$x_{1}\in S$ it's the case that $x_{1}+s_{2}\in S_{1}$ and $x_{2}+s_{2}\in S_{1}$.
Now consider $\lambda x_{1}+\left(1-\lambda\right)x_{2}$. Since $S_{1}$
is convex $\lambda\left(x_{1}+s_{2}\right)+\left(1-\lambda\right)\left(x_{2}+s_{2}\right)\in S_{1}$
but 
\begin{align*}
\lambda\left(x_{1}+s_{2}\right)+\left(1-\lambda\right)\left(x_{2}+s_{2}\right) & =\left(\lambda x_{1}+\left(1-\lambda\right)x_{2}\right)+\lambda s_{2}+\left(1-\lambda\right)s_{2}\\
 & =\left(\lambda x_{1}+\left(1-\lambda\right)x_{2}\right)+s_{2}
\end{align*}
and therefore $S$ is convex.
\item [1.3]The set is convex because it's actually the interior of a sphere.
Fix $\theta$ and without loss of generality set $a=0$ and $b'=\left(0,\dots,\left\Vert b-a\right\Vert \right)\in\mathbb{R}^{n}$
and consider the set of $x\in\mathbb{R}^{n}$ where 
\[
\left\Vert x\right\Vert _{2}=\theta\left\Vert x-b'\right\Vert _{2}
\]
Since distances are positive this itself is equivalent to 
\begin{align*}
\sum_{i=1}^{n}x_{i}^{2} & =\theta^{2}\left(\sum_{i=1}^{n-1}x_{i}^{2}+\left(x_{n}-\left\Vert b-a\right\Vert \right)^{2}\right)\\
 & =\theta^{2}\left(\sum_{i=1}^{n-1}x_{i}^{2}+x_{n}^{2}-2x_{n}\left\Vert b-a\right\Vert +\left\Vert b-a\right\Vert ^{2}\right)\\
 & =\theta^{2}\sum_{i=1}^{n}x_{i}^{2}-2x_{n}\theta^{2}\left\Vert b-a\right\Vert +\theta^{2}\left\Vert b-a\right\Vert ^{2}
\end{align*}
or
\begin{align*}
\sum_{i=1}^{n-1}x_{i}^{2}+x_{n}^{2}+2x_{n}\frac{\theta^{2}\left\Vert b-a\right\Vert }{\left(1-\theta^{2}\right)}-\frac{\theta^{2}\left\Vert b-a\right\Vert ^{2}}{\left(1-\theta^{2}\right)} & =0
\end{align*}
or 
\begin{align*}
\sum_{i=1}^{n-1}x_{i}^{2}+x_{n}^{2}-2x_{n}\left(-\frac{\theta^{2}\left\Vert b-a\right\Vert }{\left(1-\theta^{2}\right)}\right)+\left(-\frac{\theta^{2}\left\Vert b-a\right\Vert }{\left(1-\theta^{2}\right)}\right)^{2}-\left(-\frac{\theta^{2}\left\Vert b-a\right\Vert }{\left(1-\theta^{2}\right)}\right)^{2}-\frac{\theta^{2}\left\Vert b-a\right\Vert ^{2}}{\left(1-\theta^{2}\right)} & =0
\end{align*}
or
\begin{align*}
\sum_{i=1}^{n-1}x_{i}^{2}+\left(x_{n}-\left(-\frac{\theta^{2}\left\Vert b-a\right\Vert }{\left(1-\theta^{2}\right)}\right)\right)^{2} & =\left(\frac{\theta^{2}\left\Vert b-a\right\Vert }{\left(1-\theta^{2}\right)}\right)^{2}+\frac{\theta^{2}\left\Vert b-a\right\Vert ^{2}}{\left(1-\theta^{2}\right)}
\end{align*}
Therefore the boundary of the set is a sphere and since the center
of the sphere $\left(0,\dots,-\frac{\theta^{2}\left\Vert b-a\right\Vert }{\left(1-\theta^{2}\right)}\right)$
satisfies the inequality $\left\Vert x\right\Vert _{2}\leq\theta\left\Vert x-b'\right\Vert _{2}$
the set must be the entire sphere, and hence convex.
\item [1.4]The set is convex. Let $S\subset\mathbb{R}^{n}$ be the set
\[
\left\{ x\in\mathbb{R}^{n}\big|x^{\intercal}Ax+b^{\intercal}x+c\leq0,g^{\intercal}x+h=0\right\} 
\]
Since $A+\lambda gg^{\intercal}$ is PSD it's the case that 
\[
S'=\left\{ x\in\mathbb{R}^{n}\big|x^{\intercal}\left(A+\lambda gg^{\intercal}\right)x+b^{\intercal}x+c\leq0\right\} 
\]
is convex. Why? Diagonalize and complete the square to get the constraint
into the form $x^{\intercal}Bx\leq k$ (note that $B\succeq0$) and
some $k$. If $k\geq0$ then the set is convex. If $k<0$ then the
set is empty and trivially convex. Then manipulating the constraint
characterizing $S'$ 
\begin{align*}
x^{\intercal}\left(A+\lambda gg^{\intercal}\right)x+b^{\intercal}x+c & =x^{\intercal}Ax+x^{\intercal}\lambda gg^{\intercal}x+b^{\intercal}x+c\\
 & =x^{\intercal}Ax+\lambda\left\Vert g^{\intercal}x\right\Vert ^{2}+b^{\intercal}+c\\
 & =x^{\intercal}Ax+b^{\intercal}+c+\lambda h^{2}
\end{align*}
and therefore 
\[
x^{\intercal}\left(A+\lambda gg^{\intercal}\right)x+b^{\intercal}+c-\lambda h^{2}=x^{\intercal}Ax+b^{\intercal}+c
\]
but again since $A+\lambda gg^{\intercal}$ is PSD 
\[
\left\{ x\in\mathbb{R}^{n}\big|x^{\intercal}\left(A+\lambda gg^{\intercal}\right)x+b^{\intercal}+c-\lambda h^{2}\leq-\lambda h^{2}\right\} 
\]
is also convex (by the same argument as before except with $k'=k-\lambda h^{2}$).
Hence 
\[
\left\{ x\in\mathbb{R}^{n}\big|x^{\intercal}Ax+b^{\intercal}+c\leq-\lambda h^{2}\right\} 
\]
is convex and since the intersection of two convex sets (the plane
$g^{\intercal}x+h=0$ is convex) is convex the set $S$ is convex.
\item [1.5]The set is convex. Let 
\[
S=\left\{ y\in\mathbb{R}^{n}\big|\sum y_{i}B_{i}-A\succeq0\right\} 
\]
for $A,B_{i}$ symmetric matrices in $\mathbb{R}^{p\times p}$. Assume
$y_{1},y_{2}\in S$. Then 
\begin{align*}
x^{\intercal}\left[\left(\lambda y_{1}+\left(1-\lambda\right)y_{2}\right)\cdot\mathbf{B}-A\right]x & =x^{\intercal}\left[\left(\lambda y_{1}+\left(1-\lambda\right)y_{2}\right)\cdot\mathbf{B}-\lambda A-\left(1-\lambda\right)A\right]x\\
 & =x^{\intercal}\left[\lambda y_{1}\cdot\mathbf{B}-\lambda A+\left(1-\lambda\right)y_{2}\cdot\mathbf{B}-\left(1-\lambda\right)A\right]x\\
 & =\lambda x^{\intercal}\left(y_{1}\cdot\mathbf{B}-A\right)x+\left(1-\lambda\right)x^{\intercal}\left(y_{2}\cdot\mathbf{B}-A\right)x
\end{align*}
and since both $\lambda\geq0$ and $\left(1-\lambda\right)\geq0$
and both $y_{1},y_{2}$ are such that $\left(y_{1}\cdot\mathbf{B}-A\right)\succeq0$
and $\left(y_{2}\cdot\mathbf{B}-A\right)\succeq0$ it's the case that
\begin{align*}
x^{\intercal}\left[\left(\lambda y_{1}+\left(1-\lambda\right)y_{2}\right)\cdot\mathbf{B}-A\right]x & =\lambda x^{\intercal}\left(y_{1}\cdot\mathbf{B}-A\right)x+\left(1-\lambda\right)x^{\intercal}\left(y_{2}\cdot\mathbf{B}-A\right)x\\
 & \geq0
\end{align*}

\item [2.1]Claim: the conic hull $\text{coni}\left(C\right)$ of a convex
set $C$ satisfies
\[
\text{coni}\left(C\right)=\bigcup_{x\in C}\left\{ \gamma x\big|\gamma\geq0\right\} 
\]


\begin{proof}
Let $y\in\text{coni}\left(C\right)$. Then $y=\sum_{i=1}^{m}\alpha_{i}x_{i}$
where $\alpha_{i}\geq0$ and $x_{i}\in C$. If $\alpha_{i}=0$ for
all $i$ then $y=0$ immediately in $\bigcup_{x\in C}\left\{ \gamma x\big|x\in C,\,\gamma\geq0\right\} $.
Therefore assume not. Then dividing by $\sum\alpha_{i}$ we get that
\[
y^{'}\coloneqq\frac{1}{\sum_{i}\alpha}y=\sum_{i=1}^{m}\alpha_{i}^{'}x_{i}
\]
is a convex combination of the $x_{i}$ and therefore $y^{'}\in C$
(since $C$ is convex) and hence $y\in\left\{ \gamma y^{'}\big|\gamma\geq0\right\} $
and hence $y\in\bigcup_{x\in C}\left\{ \gamma x\big|\gamma\geq0\right\} $.
For the converse assume $y\in\bigcup_{x\in C}\left\{ \gamma x\big|\gamma\geq0\right\} $.
Then $y=\gamma x$ for some $x\in C$ and $\gamma\geq0$. But this
is a trivial conic combination of $x\in C$ and hence $y\in\text{coni}\left(C\right)$.
\end{proof}
\item [2.2]Claim: For $C_{1},C_{2}$ convex cones such that $0\in C_{1}\cap C_{2}$

\begin{enumerate}
\item $C_{1}\oplus C_{2}=\text{conv}\left(C_{1}\cup C_{2}\right)$\end{enumerate}
\begin{proof}
Since $C_{1}\oplus C_{2}$ is convex and contains $C_{1}\cup C_{2}$
($x+0=x\in C_{1}\oplus C_{2}$ for each $x\in C_{1}$ and $0+y=y\in C_{1}\oplus C_{2}$
for each $y\in C_{2}$) and $\text{conv}\left(C_{1}\cup C_{2}\right)$
is the smallest convex set that contains $C_{1}\cup C_{2}$ we have
that $\text{conv}\left(C_{1}\cup C_{2}\right)\subset C_{1}\oplus C_{2}$.
Conversely let $x\in C_{1}\oplus C_{2}$. Then $x=y+z$ for $y\in C_{1}$
and $z\in C_{2}$. But since $C_{1}$and $C_{2}$ are convex cones
$y^{'}\coloneqq2y\in C_{1}$ and $z^{'}\coloneqq2z\in C_{2}$ and
so 
\[
x=\frac{1}{2}y^{'}+\frac{1}{2}z^{'}\in\text{conv}\left(C_{1}\cup C_{2}\right)
\]
\end{proof}
\begin{enumerate}
\item [(b)]$C_{1}\cap C_{2}=\bigcup_{\alpha\in\left[0,1\right]}\left[\left(\alpha C_{1}\right)\cap\left(\left(1-\alpha\right)C_{2}\right)\right]$\end{enumerate}
\begin{proof}
Let $z\in C_{1}\cap C_{2}$ and then $z\in C_{1}$ and $z\in C_{2}$
and so trivially $z=\alpha z_{1}+\left(1-\alpha\right)z_{2}$ where
$z_{1}\coloneqq z\in C_{1}$ and $z_{2}\coloneqq z\in C_{2}$for any
$\alpha\in\left[0,1\right]$. Therefore $z\in\bigcup_{\alpha\in\left[0,1\right]}\left[\left(\alpha C_{1}\right)\cap\left(\left(1-\alpha\right)C_{2}\right)\right]$.
Let $z\in\bigcup_{\alpha\in\left[0,1\right]}\left[\left(\alpha C_{1}\right)\cap\left(\left(1-\alpha\right)C_{2}\right)\right]$.
If $\alpha\in\left\{ 0,1\right\} $ then $z\in C_{1}\cap\left\{ 0\right\} $
or $z\in\left\{ 0\right\} \cap C_{2}$ and so $z=0$ and since $0\in C_{1}\cap C_{2}$
it's immediate that $z\in C_{1}\cap C_{2}$. Assume $\alpha\in\left(0,1\right)$
and then $z=\alpha x_{1}$ for some $x_{1}\in C_{1}$ and $z=\left(1-\alpha\right)x_{2}$
and $x_{2}\in C_{2}$. Since $C_{1},C_{2}$ are cones $\alpha x_{1}\in C_{1}$
and $\left(1-\alpha\right)x_{2}\in C_{2}$. Therefore $z\in C_{1}$
and $z\in C_{2}$, i.e. $z\in C_{1}\cap C_{2}$. 
\end{proof}
\item [3.1]Let $A=\left\{ \left(x,y\right)\big|y\geq-1/x,x<0\right\} $
and $B=\left\{ \left(x,y\right)\big|y\geq1/x,x>0\right\} $. Both
sets are closed because they're the epigraphs of continuous convex
functions and $d\left(A,B\right)=0$ because for any $\epsilon$ 
\[
\left\Vert \left(-\frac{1}{2\left(1/\epsilon\right)},\frac{\epsilon}{2}\right)-\left(\frac{1}{2\left(1/\epsilon\right)},\frac{\epsilon}{2}\right)\right\Vert =\left\Vert \left(-\frac{2}{2\left(1/\epsilon\right)},0\right)\right\Vert =\epsilon
\]

\item [3.2]Since $v=p_{K}0$ it's the case that $v$ is the vector in $K=B-A$
that is closest to 0. Since $K$ is linear vector space (being a set
of displacements) $v$ is therefore the vector in $K$ of smallest
norm. Since $A$ is the $\infty$-norm ball (a square) shifted 2 units
to the left and $B$ is the same ball shifted 2 units to the right
$v$ must be $\left(2,0\right)$ (the smallest displacement from $B$
to $A$). Furthermore $E$ is the set of points in $A$ closest to
$B$ and hence it's the closed line segment from $\left(-1,1\right)$
to $\left(-1,-1\right)$ and similarly $F$ is the closed line segment
from $\left(1,1\right)$ to $\left(1,-1\right)$. Since $v$ is the
displacement of $B$ from $A$ it must be the case that $B-v$ is
$B$ translated in the direction $A$ the minimum such that they're
in contact. Therefore $A\cap\left(B-v\right)$ is the closed line
segment from $\left(-1,1\right)$ to $\left(-1,-1\right)$. Any point
fixed by $p_{A}p_{B}$ is in $A$ (since $p_{A}x\in A$) and therefore
any point fixed by $p_{A}p_{B}$ is a point in $A$ such that the
nearest point to the point it is nearest to in $B$, is itself. Therefore
$\text{Fix}\left(p_{A}p_{B}\right)$ is the closed line segment from
$\left(-1,1\right)$ to $\left(-1,-1\right)$.
\item [3.3]$v\coloneqq p_{K}0$ is the vector in $K=B-A$ that is ``closest''
to 0, i.e. that for which its norm is minimal. Since $A,B$ are closed
sets (infima are contained in the sets) there exists a $b^{*}\in B$
such that 
\[
\left\Vert a-b^{*}\right\Vert _{2}=\inf_{b\in B}\left\Vert a-b\right\Vert _{2}
\]
and similarly there exists a $a^{*}\in A$ such that
\[
\left\Vert a^{*}-b^{*}\right\Vert _{2}=\inf_{a\in A}\left\Vert a-b^{*}\right\Vert _{2}
\]
Hence $k\coloneqq b^{*}-a^{*}\in K$ and by definition $\left\Vert k\right\Vert _{2}\leq\left\Vert k^{'}\right\Vert _{2}$
for all $k^{'}\in K$. Therefore $d\left(A,B\right)=\left\Vert v\right\Vert _{2}$.
\item [3.4]First

\begin{lem*}
$v=f-e$ where $f\in F$ and $e\in E$ for some $e,f$. \end{lem*}
\begin{proof}
Suppose $v=b-a$ where $b\notin F$. Then $d\left(b,A\right)>d\left(A,B\right)$
and so by (3.3) 
\[
\left\Vert v\right\Vert _{2}=d\left(A,B\right)<d\left(b,A\right)=\inf_{a^{'}\in A}\left(\left\Vert b-a^{'}\right\Vert _{2}\right)\leq\left\Vert b-a\right\Vert _{2}=\left\Vert v\right\Vert _{2}
\]
a contradiction. Now suppose $a\notin E$. Then $d\left(a,B\right)>d\left(A,B\right)$
and similarly 
\[
\left\Vert v\right\Vert _{2}=d\left(A,B\right)<d\left(a,B\right)=\inf_{b^{'}\in A}\left(\left\Vert b^{'}-a\right\Vert _{2}\right)\leq\left\Vert b-a\right\Vert _{2}=\left\Vert v\right\Vert _{2}
\]
a contradiction yet again.\end{proof}
\begin{cor*}
$E+v=F$. \end{cor*}
\begin{proof}
Fix $e^{'}\in E$. Then there exists some $v^{'}$ such that $e^{'}+v^{'}=f^{'}$
for some $v^{'}$ and $f^{'}\in F$ (by definition of $E$, i.e. $d\left(e,B\right)=\left\Vert v\right\Vert _{2}$).
Suppose $v^{'}\neq v$. By the lemma, $v=f-e$ for some $e\in E$
and $f\in F$, we have that $v\in K$. Since the Minkowski difference
is convex it's the case that $\frac{1}{2}\left(v^{'}+v\right)\in K$.
But then 
\begin{align*}
\left\Vert \frac{1}{2}\left(v^{'}+v\right)\right\Vert ^{2} & =\frac{1}{4}\left(\left\Vert v^{'}\right\Vert ^{2}+\left\Vert v\right\Vert ^{2}+2v\cdot v^{'}\right)
\end{align*}
If $2v\cdot v^{'}\leq0$ then immediately we have then immediately
we have that $\left\Vert \frac{1}{2}\left(v^{'}+v\right)\right\Vert <\left\Vert v\right\Vert $
which implies there exist $e^{''}\in E$ and $f^{''}\in F$ such that
$\left\Vert f^{''}-e^{''}\right\Vert _{2}<d\left(A,B\right)$ a contradiction.
Then $v\cdot v^{'}>0$ and 
\begin{align*}
\frac{1}{4}\left(\left\Vert v^{'}\right\Vert ^{2}+\left\Vert v\right\Vert ^{2}+2v\cdot v^{'}\right) & =\frac{1}{4}\left(\left\Vert v^{'}\right\Vert ^{2}+\left\Vert v\right\Vert ^{2}+2\left|v\cdot v^{'}\right|\right)\\
 & \leq\frac{1}{4}\left(\left\Vert v^{'}\right\Vert ^{2}+\left\Vert v\right\Vert ^{2}+2\left\Vert v\right\Vert \left\Vert v^{'}\right\Vert \right)
\end{align*}
If $2\left|v\cdot v^{'}\right|<2\left\Vert v\right\Vert \left\Vert v^{'}\right\Vert $
then again we have that $\left\Vert \frac{1}{2}\left(v^{'}+v\right)\right\Vert <\left\Vert v\right\Vert $
a contradiction. Then $2\left|v\cdot v^{'}\right|=2\left\Vert v\right\Vert \left\Vert v^{'}\right\Vert $.
But Cauchy-Schwarz is saturated iff $v$ and $v^{'}$ are colinear
and we've already assumed $v\cdot v^{'}>0$ and $\left\Vert v\right\Vert =\left\Vert v^{'}\right\Vert $.
Therefore $v^{'}=v$ and hence $e^{'}+v=f^{'}$. Similarly for arbitrary
$f^{'}\in F$ and $f^{'}-v^{'}=e^{'}$ with $v^{'}=-v$. Hence $E+v=F$.
\end{proof}
\item [3.5]Claim: $E=\text{Fix}\left(p_{A}p_{B}\right)=A\cap\left(B-v\right)$.

\begin{proof}
Firstly clearly $E\subset\text{Fix}\left(p_{A}p_{B}\right)$ since
$p_{B}\left(\cdot\right)\subset B$ and $p_{A}B=E$ (hence for some
$B^{'}\subset B$ it's the case that $p_{A}B^{'}\subset E$). Let
$x\in\text{Fix}\left(p_{A}p_{B}\right)$. Note that $p_{B}\left(\cdot\right)\subset B$
is the set of points $B$ ``closest'' to $\mathbb{R}^{n}\backslash B$,
i.e. the boundary of $B$. Then $p_{A}\left(p_{B}\left(\cdot\right)\right)\subset A$
is the set of points in $A$ closest the boundary of $B$, i.e. $E$.
Hence $E=\text{Fix}\left(p_{A}p_{B}\right)$. Since $F-v=E$ and otherwise
$\left(\left(B-v\right)\backslash F\right)\cap A=\emptyset$ (otherwise
if $b\in B-v\backslash F$ would be such that $d\left(b,A\right)=d\left(A,B\right)$
and $b\in F$) we have that $A\cap\left(B-v\right)=A\cap\left(F-v\right)=E\cap\left(F-v\right)=E$.
Hence $E=A\cap\left(B-v\right)$. Therefore 
\[
\text{Fix}\left(p_{A}p_{B}\right)=E=A\cap\left(B-v\right)
\]
\end{proof}
\begin{enumerate}
\item [3.6]Claim: $p_{A}f=p_{E}f=f-v$.\end{enumerate}
\begin{proof}
Let $x=p_{A}f$, the nearest point in $A$ to $f$. Since $f\in F\subset B$,
by definition of $E$, it must be the case that $x=e$ for some $e\in E$.
Then for $y=p_{E}f$ it must be the case that $x=y$ since $y$ is
the nearest point in $E$ to $f$ and $x$ is the point in all of
$A$ that's nearest to $f$ . Finally by 3.4 for arbitrary $f\in F$
there exists $e\in E$ such that $f-v=e$ and $e$ is the ``nearest''
point in $E\subset A$ and hence by definition of $p_{E}\left(\cdot\right)$
it's the case that $p_{E}f=e=f-v$ .
\end{proof}
\item [4.1]$\left\{ x\big|Ax=0\wedge x\succ0\right\} \neq\emptyset$ iff
$\left\{ x\big|A\left(x-es\right)=A\left(\left(-e\right)s\right)\wedge\left(x-es\right)\succeq0\right\} \neq\emptyset$
for $s>0$. By Farkas' lemma this is true iff $\left\{ y\big|A^{\intercal}y\succeq0\wedge\left(A\left(-e\right)s\right)^{\intercal}y<0\right\} =\emptyset$.
But $\left(A\left(-e\right)s\right)^{\intercal}y<0$ is equivalent
to $e^{\intercal}\left(A^{\intercal}y\right)>0$. So there does not
exist $y$ such that $A^{\intercal}y\succeq0$ and $\sum_{i}A_{i}\cdot y>0$.
This is true iff there does not exist $y$ such that $A^{\intercal}y\succ0$
(since otherwise $\sum_{i}A_{i}\cdot y>0$ would be satisfied). 
\item [4.2]Claim: if $K^{\intercal}=-K$ then 
\[
\left\{ x\in\mathbb{R}^{n}\big|Kx\succeq0,x\succeq0,\left(K+I\right)x\succ0\right\} \neq\emptyset
\]


\begin{proof}
Assume $K^{\intercal}=-K$. Then $\left(K+I\right)x$............
No clue?
\end{proof}
\item [5.1]Reference: Introduction to Algorithms by CLRS. The model is
minimize the maximum flow of drugs from node 1 to node 10 given the
constraint that arcs can be turned off at a maximum cost of 15 units.
The adjacency/drug-flow matrix is 
\[
D=\begin{pmatrix}\backslash & 1 & 2 & 3 & 4 & 5 & 6 & 7 & 8 & 9 &10\\
1 & 0 & 4 & 8 & 16 & 0 & 0 & 0 & 0 & 0 &0\\
2 & 0 & 0 & 0 & 9 & 0 & 0 & 9 & 0 & 0 &0\\
3 & 0 & 0 & 0 & 3 & 6 & 0 & 0 & 0 & 19 &0\\
4 & 0 & 0 & 0 & 0 & 19 & 6 & 0 & 0 & 0 &0\\
5 & 0 & 0 & 0 & 0 & 0 & 16 & 0 & 0 & 9 &0\\
6 & 0 & 0 & 0 & 0 & 0 & 0 & 6 & 0 & 16 &0\\
7 & 0 & 0 & 0 & 0 & 0 & 0 & 0 & 2 & 0 &3\\
8 & 0 & 0 & 0 & 0 & 0 & 0 & 0 & 0 & 13 &5\\
9 & 0 & 0 & 0 & 0 & 0 & 0 & 0 & 0 & 0 &20\\
10 & 0 & 0 & 0 & 0 & 0 & 0 & 0 & 0 & 0 &0
\end{pmatrix}
\]
The checkpoint weight matrix is
\[
C=\begin{pmatrix}\backslash & 1 & 2 & 3 & 4 & 5 & 6 & 7 & 8 & 9 & 10\\
1 & 0 & 11 & 3 & 17 & 0 & 0 & 0 & 0 & 0 & 0\\
2 & 0 & 0 & 0 & 7 & 0 & 0 & 9 & 0 & 0 & 0\\
3 & 0 & 0 & 0 & 8 & 12 & 0 & 0 & 0 & 20 & 0\\
4 & 0 & 0 & 0 & 0 & 6 & 6 & 0 & 0 & 0 & 0\\
5 & 0 & 0 & 0 & 0 & 0 & 7 & 0 & 0 & 6 & 0\\
6 & 0 & 0 & 0 & 0 & 0 & 0 & 2 & 0 & 11 & 0\\
7 & 0 & 0 & 0 & 0 & 0 & 0 & 0 & 8 & 0 & 17\\
8 & 0 & 0 & 0 & 0 & 0 & 0 & 0 & 0 & 3 & 4\\
9 & 0 & 0 & 0 & 0 & 0 & 0 & 0 & 0 & 0 & 20\\
10 & 0 & 0 & 0 & 0 & 0 & 0 & 0 & 0 & 0 & 0
\end{pmatrix}
\]
Let $G=\left(V,E\right)$ be the network and the flow from node $u$
to node $v$ be $f_{uv}$. Then the flow of drugs across the network
is 
\[
\left|f\right|=\sum_{v\in V}f_{1v}
\]
and optimization problem is 
\begin{align*}
\min_{\mathbf{B}} & \max_{f}\left(\left|f\right|\right)\\
\text{s.t.} & f_{uv}\leq b_{uv}D_{uv}\\
\text{s.t.} & \sum\nolimits _{\{u:(u,v)\in E\}}f_{uv}=\sum\nolimits _{\{u:(v,u)\in E\}}f_{vu}\ \forall v\in V\setminus\left\{ 1,10\right\} \\
\text{s.t.} & \sum_{k=2}^{9}b_{uv}C_{uv}\leq15
\end{align*}
where $b_{uv}=0$ if the government chooses to put a checkpoint on
edge $\left(u,v\right)$ and 1 otherwise. For the dual formulation
let a $SR$-$SI$ \emph{cut} be be a partition of the nodes such that
$1\in SR$ and $10\in SI$. The cut-set of the is the set of edges
that cross the cut, i.e. $\left\{ \left(u,v\right)\in E\,\text{s.t. }u\in SR,v\in SI\right\} $.
Alternatively use the \textbf{B }checkpoint indicators to ``turn
on and off'' the edges. The capacity of a $SR$-$SI$ \emph{cut }is
\[
c\left(SR,SI\right)=\sum_{\left(u,v\right)\in E}D_{uv}b_{uv}
\]
The dual problem is to minimize $c\left(SR,SI\right)$ over $SR$-$SI$
cuts and, i.e. 
\begin{align*}
\min_{\mathbf{B}} & \min_{SR,SI}\left(c\right)\\
\text{s.t.} & \sum_{k=2}^{9}b_{uv}C_{uv}\leq15
\end{align*}

\item [5.2]Skip
\item [6.1]Let $x_{n}$ be a sequence in $C$ such that $\left\Vert x_{n}-x\right\Vert _{2}\geq n$.
Consider $z_{n}=\frac{x_{n}-x}{\left\Vert x_{n}-x\right\Vert _{2}}$.
Then the sequence $z_{n}$ is bounded (contained in unit norm ball)
and therefore by Bolzano-Weierstrass there exists a convergent subsequence
$z_{n_{i}}$. Claim: 
\[
z=\lim_{i\rightarrow\infty}z_{n_{i}}\in R\left(x\right)
\]


\begin{proof}
Fix $\lambda\geq0$ and let $N$ be such that $\left\Vert x_{n_{i}}-x\right\Vert _{2}\geq\lambda$
for all $n_{i}\geq N$. Then since the $z_{n_{i}}$ converge to $z$
it's the case that $z_{n_{i}}\left(\lambda\right)=\lambda\frac{x_{n_{i}}-x}{\left\Vert x_{n_{i}}-x\right\Vert _{2}}$
converges to $\lambda z$. Finally since $C$ is convex we have that
\[
x+z_{n_{i}}=x+\alpha\left(x_{n_{i}}-x\right)=\alpha x_{n_{i}}+\left(1-\alpha\right)x
\]
where since $\left\Vert x_{n_{i}}-x\right\Vert _{2}\geq\lambda$ it's
the case that 
\[
1\geq\frac{\lambda}{\left\Vert x_{n_{i}}-x\right\Vert _{2}}=\alpha\geq0
\]
and so since $x+z_{n_{i}}\in C$ for all $n_{i}$ it's the case that
$\lambda z\in C$.
\end{proof}
\item [6.2]$R\left(x\right)$ is a cone since for any vector $z\in\mathbb{R}^{n}$
such that $x+\lambda z\in C$ for all $\lambda$ it's the case that
$\alpha z$ is also such a vector: just take $\lambda^{'}=\lambda/\alpha$
then $x+\lambda^{'}\left(\alpha z\right)=x+\lambda z\in C$, since
$\lambda,\alpha$ are both positive. $R\left(x\right)$ is convex
since if $z_{1},z_{2}\in R\left(x\right)$ then, since $C$ is convex
($\alpha x_{1}+\left(1-\alpha\right)x_{2}\in C$ for all $\alpha\in\left[0,1\right]$)
\begin{align*}
\alpha x_{1}+\left(1-\alpha\right)x_{2}+\lambda\left(\alpha z_{1}+\left(1-\alpha\right)z_{2}\right) & =\alpha\left(x_{1}+\lambda z_{1}\right)+\left(1-\alpha\right)\left(x_{2}+\lambda z_{2}\right)\\
 & =\alpha y_{1}+\left(1-\alpha\right)y_{2}
\end{align*}
where $y_{1},y_{2}\in C$ and again because $C$ is convex we have
that that $\alpha y_{1}+\left(1-\alpha\right)y_{2}\in C$ and so $\left(\alpha z_{1}+\left(1-\alpha\right)z_{2}\right)\in R\left(x\right)$
for all $\alpha\in\left[0,1\right]$. Finally $R\left(x\right)$ is
closed because $C$ is closed: let $z_{k}\rightarrow z$ be a convergent
sequence in $R\left(x\right)$. Fix a $\lambda$ and then for each
$k$ we have that $x_{k}=x+\lambda z_{k}\in C$. Then since $C$ is
closed $\lim_{k\rightarrow\infty}x_{k}\in C$ and hence 
\[
\lim_{k\rightarrow\infty}x_{k}=\lim_{k\rightarrow\infty}\left(x+\lambda z_{k}\right)=x+\lambda\lim_{k\rightarrow\infty}z_{k}=x+\lambda z\in C
\]

\item [6.3]Let $z\in R\left(x\right)$ for arbitrary $x\in C$ and $x\neq\bar{x}$
and such that $\left\Vert z\right\Vert _{2}=1$ and suppose $z\notin R\left(\bar{x}\right)$.
Then there exists some $\lambda$ such that $\bar{x}+\lambda z\notin C$.
Let $\lambda_{k}=k$. Then for every $k$ the point at the intersection
of the line segement from $\bar{x}$ to $x+\lambda_{k}z$ and the
line segement from $x$ to $\bar{x}+\lambda z$ is in $C$ (since
both $\bar{x}$ and $x+\lambda_{k}z$ are in $C$). Call this point
$x_{k}$. Then $\lim_{k\rightarrow\infty}x_{k}=\bar{x}+\lambda z$
and since $C$ is closed $\bar{x}+\lambda z\in C$, which is a contradiction.
So $R\left(x\right)\subset R\left(\bar{x}\right)$. By symmetry $R\left(\bar{x}\right)\subset R\left(x\right)$.
\item [6.4]Assume $C$ is bounded but there exists some $z\in\mathcal{R}$
and $x\in C$ such that $x+\lambda z\in C$ for $\lambda\geq0$. Let
$y_{\lambda}=x+\lambda z$ and then 
\[
\left\Vert y_{\lambda}\right\Vert ^{2}=\left\Vert x\right\Vert ^{2}+\lambda^{2}\left\Vert z\right\Vert ^{2}+2\lambda x\cdot z
\]
Regardless of what $x\cdot z$ is $\lim_{\lambda\rightarrow\infty}\left\Vert y_{\lambda}\right\Vert ^{2}$
doesn't exist and hence, by contradiction $C$ is unbounded. Assume
$\mathcal{R}=\left\{ 0\right\} $. Then take $x\in C$ and $z,\lambda$
such that $x+\lambda z\in C$ and $\left\Vert z\right\Vert _{2}=1$
and $\lambda$ is maximal. This can be done since $C$ is closed (so
any sequence of $x+\lambda_{i}z\in C$ for any fixed $z$ would converge
to some $\lambda$ such that $x+\lambda z\in C$) and if $\lambda$
were unbounded for any such $z$ then $z$ would be a recession direction.
Note $x+\lambda z$ is the ``farthest'' point in $C$ from $x$.
If there were some other point $y\in C$ such that $\left\Vert y-x\right\Vert _{2}>\left\Vert x+\lambda z\right\Vert _{2}$
then $z^{'}=\frac{\left(y-x\right)}{\left\Vert y-x\right\Vert _{2}}$
would have been a direction considered (since $C$ is convex and therefore
$x+\lambda\left(y-x\right)=\lambda y+\left(1-\lambda\right)x\in C$).
Therefore the closed ball $C\subset B_{x}\left(\lambda\right)$ and
hence $C$ is bounded.\end{enumerate}

\end{document}
