% --------------------------------------------------------------
% This is all preamble stuff that you don't have to worry about.
% Head down to where it says "Start here"
% --------------------------------------------------------------
 
\documentclass[12pt]{article}
 
\usepackage[margin=1in]{geometry} 
\usepackage{amsmath,amsthm,amssymb}
\usepackage{graphicx}
\usepackage{tikz}
\usetikzlibrary{arrows}
\newcommand{\N}{\mathbb{N}}
\newcommand{\Z}{\mathbb{Z}}
 
\newenvironment{theorem}[2][Theorem]{\begin{trivlist}
\item[\hskip \labelsep {\bfseries #1}\hskip \labelsep {\bfseries #2.}]}{\end{trivlist}}
\newenvironment{lemma}[2][Lemma]{\begin{trivlist}
\item[\hskip \labelsep {\bfseries #1}\hskip \labelsep {\bfseries #2.}]}{\end{trivlist}}
\newenvironment{exercise}[2][Exercise]{\begin{trivlist}
\item[\hskip \labelsep {\bfseries #1}\hskip \labelsep {\bfseries #2.}]}{\end{trivlist}}
\newenvironment{problem}[2][Problem]{\begin{trivlist}
\item[\hskip \labelsep {\bfseries #1}\hskip \labelsep {\bfseries #2.}]}{\end{trivlist}}
\newenvironment{question}[2][Question]{\begin{trivlist}
\item[\hskip \labelsep {\bfseries #1}\hskip \labelsep {\bfseries #2.}]}{\end{trivlist}}
\newenvironment{corollary}[2][Corollary]{\begin{trivlist}
\item[\hskip \labelsep {\bfseries #1}\hskip \labelsep {\bfseries #2.}]}{\end{trivlist}}
\newenvironment{solution}
  {\begin{proof}[Solution]\renewcommand{\qedsymbol}{}}
  {\end{proof}}

\usepackage{listings}
\usepackage{color}

\definecolor{mygreen}{rgb}{0,0.6,0}
\definecolor{mygray}{rgb}{0.5,0.5,0.5}
\definecolor{mymauve}{rgb}{0.58,0,0.82}

\lstset{ %
  backgroundcolor=\color{white},   % choose the background color; you must add \usepackage{color} or \usepackage{xcolor}
  basicstyle=\footnotesize,        % the size of the fonts that are used for the code
  breakatwhitespace=false,         % sets if automatic breaks should only happen at whitespace
  breaklines=true,                 % sets automatic line breaking
  captionpos=b,                    % sets the caption-position to bottom
  commentstyle=\color{mygreen},    % comment style
  deletekeywords={...},            % if you want to delete keywords from the given language
  escapeinside={\%*}{*)},          % if you want to add LaTeX within your code
  extendedchars=true,              % lets you use non-ASCII characters; for 8-bits encodings only, does not work with UTF-8
  frame=single,                    % adds a frame around the code
  keepspaces=true,                 % keeps spaces in text, useful for keeping indentation of code (possibly needs columns=flexible)
  keywordstyle=\color{blue},       % keyword style
  language=Python,                 % the language of the code
  morekeywords={*,...},            % if you want to add more keywords to the set
  numbers=left,                    % where to put the line-numbers; possible values are (none, left, right)
  numbersep=5pt,                   % how far the line-numbers are from the code
  numberstyle=\tiny\color{mygray}, % the style that is used for the line-numbers
  rulecolor=\color{black},         % if not set, the frame-color may be changed on line-breaks within not-black text (e.g. comments (green here))
  showspaces=false,                % show spaces everywhere adding particular underscores; it overrides 'showstringspaces'
  showstringspaces=false,          % underline spaces within strings only
  showtabs=false,                  % show tabs within strings adding particular underscores
  stepnumber=1,                    % the step between two line-numbers. If it's 1, each line will be numbered
  stringstyle=\color{mymauve},     % string literal style
  tabsize=2,                       % sets default tabsize to 2 spaces
  title=\lstname                   % show the filename of files included with \lstinputlisting; also try caption instead of title
}

\begin{document}


 
% --------------------------------------------------------------
%                         Start here
% --------------------------------------------------------------
 
\title{Test Problem}%replace X with the appropriate number
\author{Maksim Levental\\ %replace with your name
MAP 4102} %if necessary, replace with your course title
 
\maketitle
 
\begin{problem}{} %You can use theorem, exercise, problem, or question here.  Modify x.yz to be whatever number you are proving
What is the expected time until the shuffler sees the Queen of Hearts on top of the deck when there are 52 cards?

\end{problem}
 
\begin{solution}

The transition matrix is 

$$ P = \begin{pmatrix} 
\frac{1}{52} & \frac{1}{52} & \frac{1}{52} & \frac{1}{52} & \frac{1}{52} & \cdots & \frac{1}{52}& \frac{1}{52}& \frac{1}{52}\\ 
\frac{51}{52} & \frac{1}{52} & 0 & 0 & 0 && 0 & 0 & 0 \\ 
0& \frac{50}{52} & \frac{2}{52} & 0 & 0&  & 0& 0 & 0 \\ 
0 & 0 & \frac{49}{52} & \frac{3}{52} & 0 & \cdots & 0& 0 & 0 \\ 
0 & 0 & 0 & \frac{48}{52} & \frac{4}{52} & \ & 0& 0 & 0\\
& \vdots & & & &\ddots & \\
0& 0 & 0& 0  &0&&  \frac{2}{52} & \frac{50}{52}&0 \\
0& 0 & 0& 0 &0 & \cdots&0& \frac{1}{52} & \frac{51}{52} \\
\end{pmatrix}
$$

Observe that $\sum_x p(x,y) = \sum_y p(x,y) = 1$. So the chain is doubly stochastic and therefore has the unique stationary distribution that is the uniform distribution $ \pi(x) = \frac{1}{52} $ for all $1 \leqslant x \leqslant 52$. By Thm 1.22 $E_y T_y = \frac{1}{\pi_y} = 52$.



\end{solution}

\end{document}



\begin{align*}
\sum_{i=1}^{k+1}i & = \left(\sum_{i=1}^{k}i\right) +(k+1)\\ 
& = \frac{k(k+1)}{2}+k+1 & (\text{by inductive hypothesis})\\
& = \frac{k(k+1)+2(k+1)}{2}\\
& = \frac{(k+1)(k+2)}{2}\\
& = \frac{(k+1)((k+1)+1)}{2}.
\end{align*}


\end{solution}
$$
\begin{alignat*}{2}
\rho_{AA} &= \rho_{BA} \cdot p(A,B) + \rho_{CA} \cdot p(A,C) = \rho_{BA} \cdot \frac{1}{2} + \rho_{CA} \cdot \frac{1}{2} & \\
\rho_{CA} &= \rho_{AA'} \cdot p(C,A) + \rho_{BA} \cdot p(C,B) = 1 \cdot  \frac{1}{2} + \rho_{BA} \cdot \frac{1}{2} & \\
\rho_{BA} &= \rho_{DA} \cdot p(B,D) = 0 \cdot 1 = 0
\end{alignat*}
Note that $\rho_{AA}$ and $\rho_{AA'}$ are different; $\rho_{AA}$ is the probability of hitting $A$ for some time $n\geq 1$ and $\rho_{AA'}$ is the probability of hitting $A$ for sometime $n \geq i$ given that $X_i = A$. Hence
\begin{alignat*}{2}
\rho_{CA} &= 1 \cdot \frac{1}{2} + 0 \cdot \frac{1}{2} = \frac{1}{2} \\
\rho_{AA} &= 0 \cdot \frac{1}{2} + \frac{1}{2} \cdot \frac{1}{2} = \frac{1}{4} \\
\end{alignat*}
